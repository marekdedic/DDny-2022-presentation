\documentclass[10pt]{beamer}

\usepackage{polyglossia}
\usepackage{csquotes}
\usepackage{bm}
\usepackage{datetime}
\usepackage{fontspec}
\usepackage{microtype}
\usepackage{color}
\usepackage{url}
\usepackage{pgfplots}
\usepackage{hyperref}
\usepackage{amsfonts}
\usepackage{amsmath}
\usepackage{amsthm}
\usepackage{subcaption}
\usepackage[backend=biber,style=iso-authoryear,sortlocale=en_US,autolang=other,bibencoding=UTF8]{biblatex}
\usepackage{booktabs}
\usepackage{graphics}
\usepackage{pifont}
\usepackage{ulem}
\usepackage{tikz}

\usepgfplotslibrary{fillbetween}
\usetikzlibrary{patterns}

\addbibresource{zotero.bib}

\setdefaultlanguage{english}
\setmainfont{TeX Gyre Termes}
\usetheme{Boadilla}
\usecolortheme{crane}
\setbeamertemplate{title page}[default][rounded=true,shadow=false]
\setbeamertemplate{section in toc}[ball unnumbered]
\setbeamertemplate{bibliography item}{}

\hypersetup{
	pdfencoding=auto,
	unicode=true,
	citecolor=green,
	filecolor=blue,
	linkcolor=red,
	urlcolor=blue
}

\makeatletter
\newcommand*{\currentSection}{\@currentlabelname}
\makeatother

\newcommand{\mathmat}{\ensuremath{\mathbf}}

\title[DDny 2022]
{
	Scalable Graph Size Reduction for Efficient GNN Application
}

\newdate{presentation}{11}{11}{2022}
\date[November 2022]{Doktorandské dny KM FJFI, \displaydate{presentation}}

\author[Procházka et al.]{
	Pavel Procházka\inst{1} \and
	Michal Mareš\inst{1,2} \and
	\underline{Marek Dědič}\inst{1,2}
}
\institute[Cisco \& CTU]{
	\inst{1}Cisco Systems, Inc. \and
	\inst{2}Czech Technical University in Prague
}

% Title card
\AtBeginSection[]{
	\begin{frame}
	\vfill
	\centering
	\begin{beamercolorbox}[sep = 8pt,center,shadow = true,rounded = true]{title}
		\usebeamerfont{title}\insertsectionhead\par%
	\end{beamercolorbox}
	\vfill
\end{frame}
}

\begin{document}

\begin{frame}
	\titlepage
\end{frame}

\begin{frame}{Motivation}
	\begin{itemize}
	    \item In Cisco, we perform node classification on large graphs to identify malicious domains
	    \begin{itemize}
	        \item Currently simple, scalable models
	    \end{itemize}
	    \item State of the art: GNNs
	    \begin{itemize}
	        \item Powerful but computationally demanding
	        \item Do not scale to our data (at reasonable cost)
	    \end{itemize}
	    \item Suggested solution: compressing the input graph
    \end{itemize}
\end{frame}

\begin{frame}{Main idea}
    \begin{tikzpicture}
\tikzset{classA/.style={circle, draw=black, fill=blue!20!yellow}, node distance=1.5cm}
\tikzset{classB/.style={circle, draw=black, fill=blue!20}, node distance=1.5cm}
\tikzset{class0/.style={circle, draw=black, fill=white}, node distance=1.5cm}
  \begin{scope}
    \node[class0] (A) {$\nu_{A}$};
    \node[class0, below right of=A] (B) {$\nu_{B}$};
    \node[class0, below right of=B] (C) {$\nu_{C}$};
    \node[class0, above right of=B] (D) {$\nu_{D}$};
    \node[class0, below right of=D] (E) {$\nu_{E}$};
    \draw [-] (A) -- (B) ;
    \draw [-] (A) -- (D) ;
    \draw [-] (C) -- (D) ;
    \draw [-] (D) -- (E) ; 
    \draw [-] (C) -- (E) ;  
    \draw [-] (C) -- (B) ;    
    \draw [-] (D) -- (B) ;
    \draw[->, thick] ([xshift=6ex] E.center) -- ([xshift=15ex] E.center) node [pos=0.5, above] (oGNN) {GNN}; \
    \node[below of=oGNN, node distance=3ex]{\alert{infeasible}};
    \uncover<2->{
        \draw[->, thick] ([yshift=-4ex] C.center) -- ([yshift=-10ex] C.center) node [pos=0.5, right] (ocoars) {Coarsening};
    }
  \end{scope}

  \begin{scope}[xshift=7.2cm, yshift=0cm]
    \node[classA] (A) {$\nu_{A}$};
    \node[classB, below right of=A] (B) {$\nu_{B}$};
    \node[classB, below right of=B] (C) {$\nu_{C}$};
    \node[classB, above right of=B] (D) {$\nu_{D}$};
    \node[classB, below right of=D] (E) {$\nu_{E}$};
    \draw [-] (A) -- (B) ;
    \draw [-] (A) -- (D) ;
    \draw [-] (C) -- (D) ;
    \draw [-] (D) -- (E) ; 
    \draw [-] (C) -- (E) ;  
    \draw [-] (C) -- (B) ;    
    \draw [-] (D) -- (B) ;   
    % \uncover<3->{\node[above of=D, node distance=4ex]{{\color{green} good performance}};}    
  \end{scope} 

\uncover<2->{
    \begin{scope}[yshift=-4.75cm, xshift=0.1cm]
        \node[class0] (AB) {$\nu_{AB}$};
    \node[class0, right of=AB] (CD) {$\nu_{CD}$};
    \node[class0, right of=CD] (E) {$\nu_{E}$};
    \draw [-] (AB) -- (CD);
    \draw [-] (CD) -- (E); 
    \uncover<3->{
        \draw[->, thick] ([xshift=6ex] E.center) -- ([xshift=15ex] E.center) node [pos=0.5, above] (oGNN) {GNN};
    }
    % \uncover<2->{\node[below of=oGNN, node distance=3ex]{{\color{green} simplified}};}
  \end{scope}
  }
  
  \uncover<3->{
    \begin{scope}[yshift=-4.75cm, xshift=8cm]
        \node[classA] (AB) {$\nu_{AB}$};
    \node[classB, right of=AB] (CD) {$\nu_{CD}$};
    \node[classB, right of=CD] (E) {$\nu_{E}$};
    \draw [-] (AB) -- (CD);
    \draw [-] (CD) -- (E);
    % \uncover<3->{\node[below of=CD, node distance=4ex]{{\color{red} worse performance}};}
    \end{scope}
}
 \end{tikzpicture}

\end{frame}

\begin{frame}{Details}
    \begin{itemize}
        \item Base model $\to$ posterior for all nodes
        \item Similarity measure
        \item Edge contraction $\to$ graph sequence
        \item GNN on each graph with label propagation
    \end{itemize}
    \vspace{1cm}
    \scalebox{0.75}{\begin{tikzpicture}
\tikzset{classA/.style={circle, draw=black, fill=blue!20!yellow}, node distance=1.5cm}
\tikzset{classB/.style={circle, draw=black, fill=blue!20}, node distance=1.5cm}
\tikzset{classX/.style={circle, draw=black, fill=red!20}, node distance=1.5cm}
  \begin{scope}
    \node[classA] (A) {$\nu_{A}$};
    \node[classA, below right of=A] (B) {$\nu_{B}$};
    \node[classB, below right of=B] (C) {$\nu_{C}$};
    \node[classB, above right of=B] (D) {$\nu_{D}$};
    \node[classB, below right of=D] (E) {$\nu_{E}$};
    \draw [-] (A) -- (B) node [midway, fill=white, inner sep=0px] {1};
    \draw [-] (A) -- (D) node [midway, fill=white, inner sep=0px] {5};
    \draw [-] (C) -- (D) node [midway, fill=white, inner sep=0px] {2};
    \draw [-] (D) -- (E) node [midway, fill=white, inner sep=0px] {3}; 
    \draw [-] (C) -- (E) node [midway, fill=white, inner sep=0px] {4};  
    \draw [-] (C) -- (B) node [midway, fill=white, inner sep=0px] {6};    
    \draw [-] (D) -- (B) node [midway, fill=white, inner sep=0px] {7};
    \draw[->, thick] ([xshift=3ex] E.center) -- ([xshift=5ex] E.center);
    \node [below of=C, node distance=4ex] {$G_1$};
  \end{scope}
  
  \begin{scope}
  [xshift=3.8cm]
    \node (A) {};
    \node[classA , below right of=A] (AB) {$\nu_{AB}$};
    \node[classB, below right of=AB] (C) {$\nu_{C}$};
    \node[classB, above right of=AB] (D) {$\nu_{D}$};
    \node[classB, below right of=D] (E) {$\nu_{E}$};
    \draw [-] (AB) -- (D) node [midway, fill=white, inner sep=0px] {5};
    \draw [-] (C) -- (D) node [midway, fill=white, inner sep=0px] {2};
    \draw [-] (D) -- (E) node [midway, fill=white, inner sep=0px] {3}; 
    \draw [-] (C) -- (E) node [midway, fill=white, inner sep=0px] {4};  
    \draw [-] (C) -- (AB) node [midway, fill=white, inner sep=0px] {6}; 
    \draw[->, thick] ([xshift=3ex] E.center) -- ([xshift=5ex] E.center);
    \node [below of=C, node distance=4ex] {$G_2$};
  \end{scope}  
  
  \begin{scope}
  [xshift=8.8cm]
    \node[classA] (AB) {$\nu_{AB}$};
    \node[classB, below of=AB, xshift=-2ex] (CD) {$\nu_{CD}$};
    \node[classB, below right of=AB] (E) {$\nu_{E}$};
    \draw [-] (AB) -- (CD) node [midway, fill=white, inner sep=0px] {5};
    \draw [-] (CD) -- (E) node [midway, fill=white, inner sep=0px] {3}; 
    \draw[->, thick] ([xshift=3ex] E.center) -- ([xshift=5ex] E.center);
    \node [below of=CD, node distance=7ex] {$G_3$};
  \end{scope} 
  
  \begin{scope}
  [xshift=11.4cm, yshift=-0.3cm]
    \node[classA] (AB) {$\nu_{AB}$};
    \node[classB, below right of=AB] (CDE) {$\nu_{CDE}$};
    \draw [-] (AB) -- (CDE) node [midway, fill=white, inner sep=0px] {5};
    \draw[->, thick] ([xshift=5ex, yshift=0.3cm] CDE.center) -- ([xshift=7ex, yshift=0.3cm] CDE.center);
    \node [below of=CDE, node distance=8ex] {$G_4$};
  \end{scope}    
  
   \begin{scope}
  [xshift=14.9cm, yshift=-1cm]
    \node[classX] (ABCDE) {$\nu_{ABCDE}$};
     \node [below of=ABCDE, node distance=10ex] {$G_5$};
  \end{scope}
\end{tikzpicture}
}
\end{frame}

\begin{frame}{Complexity performance trade-off}
	\begin{tikzpicture}[xscale=1.2]
\tikzset{point/.style={circle, fill=black, draw=black, inner sep=1pt, minimum size=1ex}}
\tikzset{b/.style={rectangle, fill=black, draw=black, inner sep=1pt, minimum size=1ex}}
\draw[->] (0,0) -- (7,0) node [pos=1, yshift=-2ex,, fill=white, inner sep=2px] {$C$};
\draw[->] (0,0) -- (0,5) node [pos=1, xshift=-2ex, fill=white, inner sep=2px] {$P$};

\node[point] at (1,1) (p1) {};
\node[above of=p1, node distance=3ex] {$G_5$};
\node[point] at (3,2) (p2) {}; \node[above of=p2, node distance=3ex] {$G_4$};
\node[point] at (4,3) (p3) {};
\node[above of=p3, node distance=3ex] {$G_3$};
\node[point] at (5,3.5) (p4) {};
\node[above of=p4, node distance=3ex] {$G_2$};
\node[point] at (6,4) (p5) {};
\node[above of=p5, node distance=3ex] {$G_1$};
\uncover<2>{
\draw[ultra thick] (4.5, 0.1) -- (4.5, -0.1) node [pos=1, yshift=-1.5ex] (CM) {$C_m$};
\draw[->, ultra thick, shorten >=0.5ex, color=blue] (4.5, 0) -- (4.5, 3.25);
\draw[->, ultra thick,shorten >=0.5ex, color=blue] (4.5, 3.25) -- (4, 3);
\draw[->, ultra thick,shorten >=0.5ex, color=blue] (4, 3) -- (0.1, 3);
\draw[ultra thick] (0.1, 3) -- (-0.1, 3) node [pos=1, xshift=-1.5ex] {$P_3$};
\node[below of=CM, node distance=4ex, color=blue] {Achieving best performance $P_3$ for complexity $C_m$.};
}
\uncover<3>{
\draw[ultra thick] (0.1, 1.5) -- (-0.1, 1.5) node [pos=1, xshift=-1.5ex] {$P_m$};
\draw[->, shorten >=0.5ex,
ultra thick,color=green!50!black] (0, 1.5) -- (2, 1.5);
\draw[->, shorten >=0.5ex, ultra thick, color=green!50!black] (2, 1.5) -- (3, 2);
\draw[->, shorten >=0.5ex,
ultra thick,color=green!50!black] (3, 2) -- (3, 0.1);
\draw[ultra thick] (3, 0.1) -- (3,-0.1) node [pos=1, yshift=-1.5ex]  {$C_4$};
\node[below of=CM, node distance=4ex, color=green!50!black] {The least complexity ($C_4$) with performance at least  $P_m$.};
}

\draw[-] (p1) -- (p2) -- (p3) -- (p4) -- (p5);
\end{tikzpicture}

\end{frame}

\begin{frame}{Complexity}
	\begin{itemize}
	    \item usually time \& memory complexity
        \item in business we can translate time to cost
        \begin{itemize}
            \item more interpretable
            \item monthly budget \$100, EKS cost \$0.33/hour
            \item therefore daily run can take 10 hours $to$ select graph size accordingly
        \end{itemize}
    \end{itemize}
\end{frame}

\begin{frame}
	\frametitle{Edge contraction induced hierarchical tree}
	\centerline{\scalebox{0.75}{\begin{tikzpicture}
\tikzset{classA/.style={circle, draw=black, fill=blue!20!yellow}, node distance=1.5cm}
\tikzset{classB/.style={circle, draw=black, fill=blue!20}, node distance=1.5cm}
\tikzset{classX/.style={circle, draw=black, fill=red!20}, node distance=1.5cm}
\uncover<5->{
  \begin{scope}
    \node[classA] (A) {$\nu_{A}$};
    \node[classA, below right of=A] (B) {$\nu_{B}$};
    \node[classB, below right of=B] (C) {$\nu_{C}$};
    \node[classB, above right of=B] (D) {$\nu_{D}$};
    \node[classB, below right of=D] (E) {$\nu_{E}$};
    \draw [-] (A) -- (B) node [midway, fill=white, inner sep=0px] {1};
    \draw [-] (A) -- (D) node [midway, fill=white, inner sep=0px] {5};
    \draw [-] (C) -- (D) node [midway, fill=white, inner sep=0px] {2};
    \draw [-] (D) -- (E) node [midway, fill=white, inner sep=0px] {3}; 
    \draw [-] (C) -- (E) node [midway, fill=white, inner sep=0px] {4};  
    \draw [-] (C) -- (B) node [midway, fill=white, inner sep=0px] {6};    
    \draw [-] (D) -- (B) node [midway, fill=white, inner sep=0px] {7};
    \draw[->, thick] ([xshift=3ex] E.center) -- ([xshift=5ex] E.center);
    \node [below of=C, node distance=4ex] {$G_1$};
  \end{scope}
  }\uncover<4->{
  \begin{scope}
  [xshift=3.8cm]
    \node (A) {};
    \node[classA , below right of=A] (AB) {$\nu_{AB}$};
    \node[classB, below right of=AB] (C) {$\nu_{C}$};
    \node[classB, above right of=AB] (D) {$\nu_{D}$};
    \node[classB, below right of=D] (E) {$\nu_{E}$};
    \draw [-] (AB) -- (D) node [midway, fill=white, inner sep=0px] {5};
    \draw [-] (C) -- (D) node [midway, fill=white, inner sep=0px] {2};
    \draw [-] (D) -- (E) node [midway, fill=white, inner sep=0px] {3}; 
    \draw [-] (C) -- (E) node [midway, fill=white, inner sep=0px] {4};  
    \draw [-] (C) -- (AB) node [midway, fill=white, inner sep=0px] {6}; 
    \draw[->, thick] ([xshift=3ex] E.center) -- ([xshift=5ex] E.center);
    \node [below of=C, node distance=4ex] {$G_2$};
  \end{scope}  
  }\uncover<3->{
  \begin{scope}
  [xshift=8.8cm]
    \node[classA] (AB) {$\nu_{AB}$};
    \node[classB, below of=AB, xshift=-2ex] (CD) {$\nu_{CD}$};
    \node[classB, below right of=AB] (E) {$\nu_{E}$};
    \draw [-] (AB) -- (CD) node [midway, fill=white, inner sep=0px] {5};
    \draw [-] (CD) -- (E) node [midway, fill=white, inner sep=0px] {3}; 
    \draw[->, thick] ([xshift=3ex] E.center) -- ([xshift=5ex] E.center);
    \node [below of=CD, node distance=7ex] {$G_3$};
  \end{scope} 
  }\uncover<2->{
  \begin{scope}
  [xshift=11.4cm, yshift=-0.3cm]
    \node[classA] (AB) {$\nu_{AB}$};
    \node[classB, below right of=AB] (CDE) {$\nu_{CDE}$};
    \draw [-] (AB) -- (CDE) node [midway, fill=white, inner sep=0px] {5};
    \draw[->, thick] ([xshift=5ex, yshift=0.3cm] CDE.center) -- ([xshift=7ex, yshift=0.3cm] CDE.center);
    \node [below of=CDE, node distance=8ex] {$G_4$};
  \end{scope}    
  
  }
   \begin{scope}
  [xshift=14.9cm, yshift=-1cm]
    \node[classX] (ABCDE) {$\nu_{ABCDE}$};
     \node [below of=ABCDE, node distance=10ex] {$G_5$};
  \end{scope}
\end{tikzpicture}
}}
	\centerline{\scalebox{0.75}{\begin{tikzpicture}
\tikzset{classA/.style={circle, draw=black, fill=blue!20!yellow}, node distance=2cm}
\tikzset{classB/.style={circle, draw=black, fill=blue!20}, node distance=2}
\tikzset{classX/.style={circle, draw=black, fill=red!20}, node distance=2cm}
\uncover<1->{
\node[classX] (ABCDE) {$\nu_{ABCDE}$};}
\uncover<2->{
\node[classA, below left of=ABCDE] (AB) {$\nu_{AB}$};
\node[classB, below right of=ABCDE] (CDE) {$\nu_{CDE}$};}
\uncover<5->{
\node[classA, below left of=AB] (A) {$\nu_A$};
\node[classA, below right of=AB, xshift=-5ex] (B) {$\nu_{B}$};}
\uncover<3->{
\node[classB, below left of=CDE, xshift=5ex] (E) {$\nu_E$};
\node[classB, below right of=CDE] (CD) {$\nu_{CD}$};}
\uncover<4->{
\node[classB, below left of=CD] (C) {$\nu_{C}$};
\node[classB, below right of=CD] (D) {$\nu_{D}$};
}
\uncover<2->{
\draw[->] (ABCDE)--(AB);
\draw[->] (ABCDE)--(CDE);
}
\uncover<5->{
\draw[->] (AB)--(A);
\draw[->] (AB)--(B);
}
\uncover<3->{
\draw[->] (CDE)--(E);
\draw[->] (CDE)--(CD);
}
\uncover<4->{
\draw[->] (CD)--(C);
\draw[->] (CD)--(D);
}
\end{tikzpicture}
}}
\end{frame}

\begin{frame}
	\frametitle{Shared prediction induced performance upper bound}
    \centering{\scalebox{1.2}{\begin{tikzpicture}
\tikzset{classA/.style={circle, draw=black, fill=blue!20!yellow}, node distance=2cm}
\tikzset{classB/.style={circle, draw=black, fill=blue!20}, node distance=2}
\tikzset{classX/.style={circle, draw=black, fill=red!20}, node distance=2cm}
\tikzset{class0/.style={circle, draw=black}, node distance=2cm}
\tikzset{graphEdge/.style={thick, draw=black}}
\tikzset{treeEdge/.style={draw=black, ->}}


\uncover<1>{\node[draw=none, inner sep=0pt] (ABCDE) {Graph $G_4$ -- the fourth step of the coarsening procedure};}
\node[class0, below left of=ABCDE] (AB) {$\nu_{AB}$};
\node[class0, below right of=ABCDE] (CDE) {$\nu_{CDE}$};
\draw[graphEdge] (AB)--(CDE);
\uncover<2>{
\node[draw=none, inner sep=0pt] (ABCDE) {A model prediction for nodes in $G_4$};
}
\uncover<2->{
\node[classA, below left of=ABCDE] {$\nu_{AB}$};
\node[classB, below right of=ABCDE] {$\nu_{CDE}$};
}
\uncover<3>{
\node[draw=none, inner sep=0pt] (ABCDE) {Hierarchical tree corresponding to nodes in $G_4$};}
\uncover<3->{
\node[class0, below left of=AB] (A) {$\nu_A$};
\node[class0, below right of=AB, xshift=-5ex] (B) {$\nu_{B}$};
\node[class0, below left of=CDE, xshift=5ex] (E) {$\nu_E$};
\node[class0, below right of=CDE] (CD) {$\nu_{CD}$};
\node[class0, below left of=CD] (C) {$\nu_{C}$};
\node[class0, below right of=CD] (D) {$\nu_{D}$};
\draw[treeEdge] (AB)--(A);
\draw[treeEdge] (AB)--(B);
\draw[treeEdge] (CDE)--(E);
\draw[treeEdge] (CDE)--(CD);
\draw[treeEdge] (CD)--(C);
\draw[treeEdge] (CD)--(D);
}

\uncover<4>{
\node[draw=none, inner sep=0pt] (ABCDE) {Label refinement into nodes from the original graph $G_1$};
}
\uncover<4->{
\node[classA, below left of=AB] {$\nu_A$};
\node[classA, below right of=AB, xshift=-5ex] {$\nu_{B}$};
\node[classB, below left of=CDE, xshift=5ex] {$\nu_E$};
\node[classB, below left of=CD] {$\nu_{C}$};
\node[classB, below right of=CD] {$\nu_{D}$};
}

\uncover<5>{
\node[draw=none, inner sep=0pt] (ABCDE) {Comparing refined prediction with true labels on $G_1 \to$ $Acc=0.8$.};
}
\uncover<5->{
 \node[below of=A, node distance=3ex, color=green] {{\large{$\bm{\surd}$}}};
 \node[below of=B, node distance=3ex, color=green] {{\large{$\bm{\surd}$}}}; 
 \node[below of=C, node distance=3ex, color=green] {{\large{$\bm{\surd}$}}}; 
  \node[below of=D, node distance=3ex, color=green] {{\large{$\bm{\surd}$}}}; 
  \node[below of=E, node distance=3ex, color=red] {{\Large{$\bm{\times}$}}};  
}
\uncover<6>{
\node[draw=none, inner sep=0pt]  {\alert{Accuracy upper-bound} for $G_4$ $\to$ Acc=0.8};
}
\uncover<7>{
\node[draw=none, inner sep=0pt]  {UB=1 until merging nodes with \alert{the same label}
};}

\end{tikzpicture}
}}
\end{frame}

\begin{frame}{Experiment Setup}
    \begin{itemize}
        \item Base model: logistic regression
        \item Similarity measure: KL divergence
        \item GNN: 2-layer GCN
        \item Dataset: Cora
    \end{itemize}
\end{frame}

\begin{frame}{Comparison of Performance Given by Probability Similarities}
    % This file was created with tikzplotlib v0.10.1.

\tikzset{
    invisible/.style={opacity=0},
    visible on/.style={alt={#1{}{invisible}}},
    alt/.code args={<#1>#2#3}{%
      \alt<#1>{\pgfkeysalso{#2}}{\pgfkeysalso{#3}} % \pgfkeysalso doesn't change the path
    },
  }
  
\begin{tikzpicture}

\definecolor{darkgray176}{RGB}{176,176,176}
\definecolor{darkorange25512714}{RGB}{255,127,14}
\definecolor{forestgreen4416044}{RGB}{44,160,44}
\definecolor{lightblue187214232}{RGB}{187,214,232}
\definecolor{lightgray191226191}{RGB}{191,226,191}
\definecolor{lightgray204}{RGB}{204,204,204}
\definecolor{peachpuff254216182}{RGB}{254,216,182}
\definecolor{steelblue31119180}{RGB}{31,119,180}

\begin{axis}[
height=0.525\linewidth,width=0.88\linewidth,
legend cell align={left},
legend style={
  fill opacity=0.8,
  draw opacity=1,
  text opacity=1,
  at={(0.03,0.97)},
  anchor=north west,
  draw=lightgray204,
  nodes={scale=0.6, transform shape}
},
tick align=outside,
tick pos=left,
title={Cora Dataset, train ratio 10\%},
x grid style={darkgray176},
xlabel={Complexity -- Number of nodes},
xmajorgrids,
xmin=-135.4, xmax=2843.4,
xtick style={color=black},
y grid style={darkgray176},
ylabel={Performance -- Test Accuracy},
ymajorgrids,
ymin=0.3, ymax=1,
ytick style={color=black}
]
% \path [draw=lightblue187214232, fill=lightblue187214232]
% (axis cs:2708,0.827451330819263)
% --(axis cs:2708,0.802415412965098)
% --(axis cs:2629,0.796108138087677)
% --(axis cs:2535,0.797290363761398)
% --(axis cs:2437,0.800838112865581)
% --(axis cs:2340,0.775643296180803)
% --(axis cs:2242,0.757187268690742)
% --(axis cs:2139,0.729373378021187)
% --(axis cs:2039,0.650226879386432)
% --(axis cs:1940,0.687591189664324)
% --(axis cs:1838,0.619945967447788)
% --(axis cs:1739,0.570780465297014)
% --(axis cs:1640,0.495217610644987)
% --(axis cs:1542,0.556101283440238)
% --(axis cs:1443,0.40739980630099)
% --(axis cs:1343,0.524473375575064)
% --(axis cs:1243,0.479073630445438)
% --(axis cs:1146,0.478305289133408)
% --(axis cs:1049,0.387911668020888)
% --(axis cs:949,0.32086497289891)
% --(axis cs:846,0.236052627128458)
% --(axis cs:748,0.18715880050073)
% --(axis cs:650,0.299224013650956)
% --(axis cs:548,0.147523483370609)
% --(axis cs:450,0.180411986745135)
% --(axis cs:360,0.261730650880642)
% --(axis cs:265,0.2708085184999)
% --(axis cs:179,0.267403432066264)
% --(axis cs:98,0.260218413977719)
% --(axis cs:24,0.256816667432336)
% --(axis cs:24,0.285482733974802)
% --(axis cs:24,0.285482733974802)
% --(axis cs:98,0.288998376992909)
% --(axis cs:179,0.296416736839462)
% --(axis cs:265,0.29992903993957)
% --(axis cs:360,0.290560535448312)
% --(axis cs:450,0.205861846163746)
% --(axis cs:548,0.171113645820647)
% --(axis cs:650,0.32915845725798)
% --(axis cs:748,0.212949812300192)
% --(axis cs:846,0.26397383781444)
% --(axis cs:949,0.351327631181599)
% --(axis cs:1049,0.419554328620181)
% --(axis cs:1146,0.510550028053023)
% --(axis cs:1243,0.511318884370462)
% --(axis cs:1343,0.556613799398024)
% --(axis cs:1443,0.439264730026702)
% --(axis cs:1542,0.588010994432195)
% --(axis cs:1640,0.527456054389835)
% --(axis cs:1739,0.602538567619249)
% --(axis cs:1838,0.650985115525478)
% --(axis cs:1940,0.717076089178487)
% --(axis cs:2039,0.680654911836205)
% --(axis cs:2139,0.75753397662631)
% --(axis cs:2242,0.784290604477592)
% --(axis cs:2340,0.801958723636254)
% --(axis cs:2437,0.825954235344569)
% --(axis cs:2535,0.822584594430551)
% --(axis cs:2629,0.821461023438615)
% --(axis cs:2708,0.827451330819263)
% --cycle;

\path [draw=peachpuff254216182, fill=peachpuff254216182]
(axis cs:2708,0.829322241283768)
--(axis cs:2708,0.804387496977853)
--(axis cs:2654,0.80004958411148)
--(axis cs:2596,0.807938552044731)
--(axis cs:2531,0.796896268837635)
--(axis cs:2460,0.803993039121293)
--(axis cs:2384,0.802021057495152)
--(axis cs:2306,0.792168654594336)
--(axis cs:2226,0.796108138087677)
--(axis cs:2143,0.798472767278663)
--(axis cs:2058,0.787837436846602)
--(axis cs:1975,0.778002102585593)
--(axis cs:1891,0.779968262716348)
--(axis cs:1803,0.763466031650242)
--(axis cs:1713,0.758756569321544)
--(axis cs:1629,0.746992999601444)
--(axis cs:1544,0.739549936375248)
--(axis cs:1456,0.723115716826995)
--(axis cs:1368,0.712954623042861)
--(axis cs:1278,0.52601468804413)
--(axis cs:1192,0.490603327961759)
--(axis cs:1104,0.524088071541332)
--(axis cs:1017,0.433040888574637)
--(axis cs:929,0.400518678008559)
--(axis cs:844,0.427679430284647)
--(axis cs:752,0.402812030300089)
--(axis cs:669,0.322005005262438)
--(axis cs:578,0.281408778502217)
--(axis cs:485,0.317065601784616)
--(axis cs:402,0.286712612963027)
--(axis cs:320,0.285954773392922)
--(axis cs:246,0.269673372571433)
--(axis cs:166,0.128918399979808)
--(axis cs:94,0.265512174907311)
--(axis cs:29,0.260974505669672)
--(axis cs:29,0.289779482980363)
--(axis cs:29,0.289779482980363)
--(axis cs:94,0.294464999809069)
--(axis cs:166,0.151288781770136)
--(axis cs:246,0.298758389357784)
--(axis cs:320,0.315526738428048)
--(axis cs:402,0.316306096522968)
--(axis cs:485,0.34744101403614)
--(axis cs:578,0.31084954732083)
--(axis cs:669,0.35249339529478)
--(axis cs:752,0.434629320224637)
--(axis cs:844,0.459720843305378)
--(axis cs:929,0.432311079613457)
--(axis cs:1017,0.465119768435348)
--(axis cs:1104,0.556230504616477)
--(axis cs:1192,0.522847151296023)
--(axis cs:1278,0.558146882190182)
--(axis cs:1368,0.741671580187193)
--(axis cs:1456,0.751494056299034)
--(axis cs:1544,0.767340988329669)
--(axis cs:1629,0.774501303295829)
--(axis cs:1713,0.785795699285707)
--(axis cs:1803,0.790309423295433)
--(axis cs:1891,0.80608834469835)
--(axis cs:1975,0.804211510462189)
--(axis cs:2058,0.813591148112732)
--(axis cs:2143,0.823707987582437)
--(axis cs:2226,0.821461023438615)
--(axis cs:2306,0.817714518070855)
--(axis cs:2384,0.827077087395031)
--(axis cs:2460,0.828948100244018)
--(axis cs:2531,0.822210090465377)
--(axis cs:2596,0.832688576303584)
--(axis cs:2654,0.825205566314038)
--(axis cs:2708,0.829322241283768)
--cycle;

% \path [draw=lightgray191226191, fill=lightgray191226191]
% (axis cs:2708,0.818838673073156)
% --(axis cs:2708,0.793350296246499)
% --(axis cs:2608,0.789018451079413)
% --(axis cs:2507,0.793350296246499)
% --(axis cs:2407,0.794532111709025)
% --(axis cs:2307,0.783901933525988)
% --(axis cs:2208,0.758756569321544)
% --(axis cs:2107,0.742291483098077)
% --(axis cs:2007,0.724679793085902)
% --(axis cs:1907,0.730547095038104)
% --(axis cs:1808,0.678239772798704)
% --(axis cs:1710,0.646729485063947)
% --(axis cs:1609,0.630420269713617)
% --(axis cs:1510,0.591665219440051)
% --(axis cs:1410,0.535651390186272)
% --(axis cs:1307,0.527556154737289)
% --(axis cs:1205,0.438787415020254)
% --(axis cs:1107,0.369212502721794)
% --(axis cs:1006,0.449903520362277)
% --(axis cs:906,0.290123500247009)
% --(axis cs:810,0.273079159571887)
% --(axis cs:712,0.22361531807037)
% --(axis cs:615,0.32618599083971)
% --(axis cs:517,0.324665475264258)
% --(axis cs:417,0.177789909534623)
% --(axis cs:320,0.193536342139995)
% --(axis cs:221,0.275728825553432)
% --(axis cs:130,0.121495739965741)
% --(axis cs:35,0.25530513225628)
% --(axis cs:35,0.283919873785997)
% --(axis cs:35,0.283919873785997)
% --(axis cs:130,0.143339462169962)
% --(axis cs:221,0.305000517752671)
% --(axis cs:320,0.219638451571034)
% --(axis cs:417,0.203103731162365)
% --(axis cs:517,0.355213117069827)
% --(axis cs:615,0.356766996794134)
% --(axis cs:712,0.251047384760021)
% --(axis cs:810,0.30226999188517)
% --(axis cs:906,0.319812598727169)
% --(axis cs:1006,0.482075484523006)
% --(axis cs:1107,0.40059215179833)
% --(axis cs:1205,0.470902224222447)
% --(axis cs:1307,0.559679810758424)
% --(axis cs:1410,0.567725150434996)
% --(axis cs:1510,0.623158149654634)
% --(axis cs:1609,0.66126298146367)
% --(axis cs:1710,0.677234916727999)
% --(axis cs:1808,0.707981134070079)
% --(axis cs:1907,0.758666056148945)
% --(axis cs:2007,0.753004375416978)
% --(axis cs:2107,0.769979633564832)
% --(axis cs:2208,0.785795699285707)
% --(axis cs:2307,0.809840662645187)
% --(axis cs:2407,0.819962654268381)
% --(axis cs:2507,0.818838673073156)
% --(axis cs:2608,0.81471593052283)
% --(axis cs:2708,0.818838673073156)
% --cycle;

% \addplot [thick, steelblue31119180]
% table {%
% 2708 0.815071126489812
% 2629 0.808919646289889
% 2535 0.810073048827374
% 2437 0.813533256439831
% 2340 0.788927335640138
% 2242 0.770857362552864
% 2139 0.743560169165705
% 2039 0.665513264129181
% 1940 0.70242214532872
% 1838 0.635524798154556
% 1739 0.586697424067666
% 1640 0.511341791618608
% 1542 0.572087658592849
% 1443 0.423298731257209
% 1343 0.540561322568243
% 1243 0.49519415609381
% 1146 0.49442522106882
% 1049 0.403690888119954
% 949 0.3360246059208
% 846 0.249903883121876
% 748 0.199923106497501
% 650 0.314109957708574
% 548 0.15916955017301
% 450 0.193002691272587
% 360 0.276047673971549
% 265 0.285274894271434
% 179 0.281814686658977
% 98 0.274509803921569
% 24 0.271049596309112
% };
% \addlegendentry{Cross Entropy}
\addplot [ultra thick, darkorange25512714]
table {%
2708 0.816993464052288
2654 0.81276432141484
2596 0.820453671664744
2531 0.809688581314879
2460 0.816608996539792
2384 0.814686658977316
2306 0.805074971164937
2226 0.808919646289889
2143 0.81122645136486
2058 0.800845828527489
1975 0.79123414071511
1891 0.793156478277586
1803 0.777008842752787
1713 0.772395232602845
1629 0.760861207227989
1544 0.753556324490581
1456 0.737408688965782
1368 0.727412533640907
1278 0.542099192618224
1192 0.506728181468666
1104 0.540176855055748
1017 0.449058054594387
929 0.416378316032295
844 0.443675509419454
752 0.418685121107266
669 0.337178008458285
578 0.296039984621299
485 0.332179930795848
402 0.301422529796232
320 0.300653594771242
246 0.284121491733948
166 0.139946174548251
94 0.279892349096501
29 0.275278738946559
};
\addlegendentry{actual}
% \addplot [ultra thick, forestgreen4416044]
% table {%
% 2708 0.806228373702422
% 2608 0.801999231064975
% 2507 0.806228373702422
% 2407 0.807381776239908
% 2307 0.797001153402538
% 2208 0.772395232602845
% 2107 0.756247597078047
% 2007 0.738946559015763
% 1907 0.744713571703191
% 1808 0.693194925028835
% 1710 0.662053056516724
% 1609 0.645905420991926
% 1510 0.607458669742407
% 1410 0.551710880430604
% 1307 0.543637062668205
% 1205 0.454825067281815
% 1107 0.384851980007689
% 1006 0.465974625144175
% 906 0.304882737408689
% 810 0.287581699346405
% 712 0.237216455209535
% 615 0.341407151095732
% 517 0.339869281045752
% 417 0.190311418685121
% 320 0.206459054209919
% 221 0.290272971933872
% 130 0.132256824298347
% 35 0.269511726259131
% };
% \addlegendentry{Random}
% \addplot [semithick, steelblue31119180]
% table {%
% 2708 1
% 2629 0.99923106497501
% 2535 0.998077662437524
% 2437 0.995386389850058
% 2340 0.985005767012687
% 2242 0.97923875432526
% 2139 0.971549404075356
% 2039 0.952710495963091
% 1940 0.934640522875817
% 1838 0.911956939638601
% 1739 0.895809304113802
% 1640 0.881968473663975
% 1542 0.85121107266436
% 1443 0.826605151864667
% 1343 0.801230296039985
% 1243 0.776239907727797
% 1146 0.75278738946559
% 1049 0.729719338715878
% 949 0.69357939254133
% 846 0.612841214917339
% 748 0.58477508650519
% 650 0.558246828143022
% 548 0.536332179930796
% 450 0.499038831218762
% 360 0.472510572856594
% 265 0.444444444444444
% 179 0.420607458669742
% 98 0.379853902345252
% 24 0.359477124183007
% };
% \addlegendentry{Cross Entropy - UB}
\addplot [semithick, darkorange25512714]
table {%
2708 1
2654 0.998462129950019
2596 0.996924259900038
2531 0.994232987312572
2460 0.993079584775087
2384 0.988850442137639
2306 0.984621299500192
2226 0.98077662437524
2143 0.977700884275279
2058 0.971933871587851
1975 0.967704728950404
1891 0.962706651287966
1803 0.958861976163014
1713 0.941945405613226
1629 0.933102652825836
1544 0.927720107650904
1456 0.913110342176086
1368 0.896578239138793
1278 0.788158400615148
1192 0.748173779315648
1104 0.730872741253364
1017 0.698961937716263
929 0.657054978854287
844 0.622837370242215
752 0.599384851980008
669 0.572856593617839
578 0.546712802768166
485 0.492887351018839
402 0.463283352556709
320 0.434832756632065
246 0.409457900807382
166 0.392541330257593
94 0.372164552095348
29 0.35678585159554
};
\addlegendentry{upper bound}
% \addplot [semithick, forestgreen4416044]
% table {%
% 2708 1
% 2608 0.995001922337562
% 2507 0.987697039600154
% 2407 0.98000768935025
% 2307 0.974240676662822
% 2208 0.960399846212995
% 2107 0.95117262591311
% 2007 0.945021145713187
% 1907 0.933487120338331
% 1808 0.915417147251057
% 1710 0.89042675893887
% 1609 0.844675124951942
% 1510 0.818146866589773
% 1410 0.782775855440215
% 1307 0.755094194540561
% 1205 0.718954248366013
% 1107 0.687427912341407
% 1006 0.655132641291811
% 906 0.601307189542484
% 810 0.577085736255286
% 712 0.541714725105729
% 615 0.512495194156094
% 517 0.488273740868897
% 417 0.459823144944252
% 320 0.434832756632065
% 221 0.399846212995002
% 130 0.3760092272203
% 35 0.361014994232987
% };
% \addlegendentry{Random - UB}


\addplot [ultra thick, black, dotted]
table {%
0 0.559822747415066
2708 0.559822747415066
};
\addlegendentry{base model}

\addplot [thick, black, dotted, name path=AA, visible on=<2>]
table {%
1278 0.559822747415066
2708 0.559822747415066
};
\addplot [thick, darkorange25512714, visible on=<2>, name path=BB]
table {%
2708 0.816993464052288
2654 0.81276432141484
2596 0.820453671664744
2531 0.809688581314879
2460 0.816608996539792
2384 0.814686658977316
2306 0.805074971164937
2226 0.808919646289889
2143 0.81122645136486
2058 0.800845828527489
1975 0.79123414071511
1891 0.793156478277586
1803 0.777008842752787
1713 0.772395232602845
1629 0.760861207227989
1544 0.753556324490581
1456 0.737408688965782
1368 0.727412533640907
1278 0.542099192618224
};
\addplot[visible on=<2>, pattern=north east lines] fill between[of=AA and BB];


% \node[visible on=<2>, fill=white] at (axis cs:2000, 0.4) {Considerable performance gain};
\end{axis}

\end{tikzpicture}

\end{frame}

\begin{frame}{Conclusions \& Future Work}
    \begin{itemize}
        \item Conclusions
        \begin{itemize}
            \item Performance-complexity trade-off
            \begin{itemize}
                \item Ability to set cost limit for running the algorithm
            \end{itemize}
            \item Hierarchical tree as a byproduct
            \item Computationally cheap upper bound
            \item Transfer learning: like using other hyperparameters
            \begin{itemize}
                \item Assuming the data does not change significantly
            \end{itemize}
        \end{itemize}
        
        \item Future Work
        \begin{itemize}
            \item Run on our datasets
            \item Use for smarter coarsening of riskmap input graph
            \begin{itemize}
                \item Current coarsening rather simple: drop nodes with deg < 3  
            \end{itemize}
        \end{itemize}
        
    \item More details can be found in the ITAT paper \cite{prochazka_scalable_2022}
    \item My related work presented at ECML \cite{dedic_adaptive_2022}
    \end{itemize}
\end{frame}

\begin{frame}
	\titlepage
\end{frame}

\end{document}
